\section{Data description}

The data for this project lies within a cloned git repository. Git stores data as a series of snapshots instead of storing differences between file versions like many other version control systems. Each commit in Git represents a snapshot of the entire working directory at a given point in time. The data is stored as objects, which are identified by a unique SHA-1 hash. There are four types of objects:
\begin{itemize}
    \item Blob: Represents a file and its contents.
    \item Tree: Represents a directory and stores references to blob objects (files) and other tree objects (subdirectories).
    \item Commit: Represents a snapshot of the repository, including metadata such as author, committer, date, and a message. A commit object also references the tree object representing the root directory of the snapshot.
    \item Tag: A lightweight object that references a commit object to create a more human-readable name for a particular commit, typically used for marking releases.
\end{itemize}
Git objects are stored in the ".git/objects" directory as individual files. However, to save space and improve performance, Git periodically repacks objects into packfiles. A packfile contains multiple objects and their delta-compressed versions, which helps reduce storage space.
Although we could write our own scripts to go through all this data stored in the .git folder, it is not necessary to do so. Git already provides a efficient command line interface to query this database. Additional details about the inner workings of git can be found at \cite{gitinternals} and the git docs \cite{gitdocs}. 

Some data does not reside locally in the cloned repository and can be accessed using Github's REST API \cite{gitrestapi}.

A general list of attributes that one can query using git/Github's REST API:
\begin{itemize}
    \item branches
    \item commits
    \item contributors
    \item pull requests
    \item forks
    \item files
    \item directories
    \item code and line count - additions/deletions
\end{itemize}

This list is not perfect as each attribute on the list can have further details such as each commit has a date associated with it and each pull request only affects the changed files and directories. The goal of this project is to use a combination of these attributes and make it easy for a developer to gain insights about the code base. For example, a developer might be interested in the pull requests that have modified a specific file.